Este trabalho apresentou o desenvolvimento de uma biblioteca em MATLAB para otimização de trajetórias, com foco em problemas de controle ótimo. A biblioteca foi implementada de forma modular e flexível, permitindo sua aplicação a diferentes tipos de problemas através de uma interface unificada.

Os resultados obtidos nos problemas exemplo demonstraram a eficácia da biblioteca em realizar a transcrição de problemas de otimização de trajetória para problemas de programação não-linear, permitindo sua solução através de métodos de otimização numérica. Quanto aos problemas propostos, obteve-se a solução ótima apenas para aqueles de menor complexidade. Para os problemas mais complexos, a solução encontrada não foi de boa qualidade, necessitando persistir na otimização dos parâmetros utilizados para o solver, ou ainda indicando falhas de implementação na biblioteca.

Algumas das principais contribuições deste trabalho incluem o desenvolvimento de uma estrutura modular e extensível para problemas de otimização de trajetória, uma interface unificada que facilita a definição e solução de novos problemas, e a validação através de problemas exemplo com baixos níveis de complexidade.

Como sugestões para trabalhos futuros, sugere-se a verificação da implementação do escalonamento das variáveis de decisão, a depuração dos problemas exemplos mais complexos e a implementação de métodos adicionais de discretização, como o de Hermite-Simpson \cite{kelly_introduction_2017,betts_practical_2010}. Além disso, propõe-se o desenvolvimento de uma interface gráfica para facilitar a definição de problemas, a aplicação da biblioteca a outros problemas práticos mais complexos e a otimização do desempenho computacional para problemas de grande escala.

Por fim, conclui-se que os objetivos iniciais do trabalho foram parcialmente alcançados, resultando em uma ferramenta que fornece uma base para o estudo e solução de problemas de otimização de trajetória. A biblioteca desenvolvida pode servir como base para futuros estudos na área, desde que sejam corrigidos os problemas encontrados.
