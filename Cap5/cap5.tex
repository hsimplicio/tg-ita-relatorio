Este trabalho apresentou o desenvolvimento de uma biblioteca em MATLAB para otimização de trajetórias, com foco em problemas de controle ótimo. A biblioteca foi implementada de forma modular e flexível, permitindo sua aplicação a diferentes tipos de problemas através de uma interface unificada.

Os resultados obtidos nos problemas exemplo demonstraram a eficácia da biblioteca em encontrar soluções ótimas, com boa concordância com resultados analíticos quando disponíveis. O problema do movimento simples em uma dimensão serviu como validação inicial da implementação, enquanto o problema clássico da braquistócrona permitiu uma comparação direta com a solução analítica conhecida. Por fim, o problema mais complexo da trajetória do eVTOL demonstrou a capacidade da biblioteca em lidar com sistemas não-lineares e restrições mais elaboradas.

Algumas das principais contribuições deste trabalho incluem:
\begin{itemize}
    \item Desenvolvimento de uma estrutura modular e extensível para problemas de otimização de trajetória
    \item Implementação de diferentes métodos de discretização e técnicas de otimização
    \item Interface unificada que facilita a definição e solução de novos problemas
    \item Validação através de problemas exemplo com diferentes níveis de complexidade
\end{itemize}

Como sugestões para trabalhos futuros, podem ser citadas:
\begin{itemize}
    \item Implementação de métodos adicionais de discretização
    \item Incorporação de técnicas mais avançadas de otimização
    \item Desenvolvimento de uma interface gráfica para facilitar a definição de problemas
    \item Aplicação da biblioteca a problemas práticos mais complexos
    \item Otimização do desempenho computacional para problemas de grande escala
\end{itemize}

Por fim, conclui-se que os objetivos iniciais do trabalho foram alcançados, resultando em uma ferramenta útil para o estudo e solução de problemas de otimização de trajetória. A biblioteca desenvolvida pode servir como base para futuros desenvolvimentos na área, tanto em contextos acadêmicos quanto em aplicações práticas.
