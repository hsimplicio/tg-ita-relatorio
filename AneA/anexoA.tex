Nessa seção, serão apresentados os arquivos de código fonte do projeto. Os mesmo arquivos estão disponíveis no repositório do projeto \cite{simplicio_hsimpliciotg-ita_2024}. Para melhor visualização, sugere-se que o leitor abra os arquivos no editor de sua preferência.

\section{Arquivos da biblioteca}
\label{sec:arquivos-biblioteca}

\subsection{\texttt{TrajectoryProblem.m}}
\label{sec:codigo-trajectoryproblem}

\begin{code}
    \inputminted[
        frame=leftline,
        framesep=3mm,
        baselinestretch=1.2,
        bgcolor=light_bg,
        rulecolor=light_rule,
        linenos
    ]{matlab}{AneA/codigos/TrajectoryProblem.m}
    \caption{Classe \texttt{TrajectoryProblem}}
    \label{code:trajectoryproblem}
\end{code}

\subsection{\texttt{packZ.m}}
\label{sec:codigo-packz}

\begin{code}
    \inputminted[
        frame=leftline,
        framesep=3mm,
        baselinestretch=1.2,
        bgcolor=light_bg,
        rulecolor=light_rule,
        linenos
    ]{matlab}{AneA/codigos/packZ.m}
    \caption{Função \texttt{packZ}}
    \label{code:packz}
\end{code}

\subsection{\texttt{unpackZ.m}}
\label{sec:codigo-unpackz}

\begin{code}
    \inputminted[
        frame=leftline,
        framesep=3mm,
        baselinestretch=1.2,
        bgcolor=light_bg,
        rulecolor=light_rule,
        linenos
    ]{matlab}{AneA/codigos/unpackZ.m}
    \caption{Função \texttt{unpackZ}}
    \label{code:unpackz}
\end{code}

\subsection{\texttt{computeDefects.m}}
\label{sec:codigo-computedefects}

\begin{code}
    \inputminted[
        frame=leftline,
        framesep=3mm,
        baselinestretch=1.2,
        bgcolor=light_bg,
        rulecolor=light_rule,
        linenos
    ]{matlab}{AneA/codigos/computeDefects.m}
    \caption{Função \texttt{computeDefects}}
    \label{code:computedefects}
\end{code}

\subsection{\texttt{evaluateConstraints.m}}
\label{sec:codigo-evaluateconstraints}

\begin{code}
    \inputminted[
        frame=leftline,
        framesep=3mm,
        baselinestretch=1.2,
        bgcolor=light_bg,
        rulecolor=light_rule,
        linenos
    ]{matlab}{AneA/codigos/evaluateConstraints.m}
    \caption{Função \texttt{evaluateConstraints}}
    \label{code:evaluateconstraints}
\end{code}

\subsection{\texttt{evaluateObjective.m}}
\label{sec:codigo-evaluateobjective}

\begin{code}
    \inputminted[
        frame=leftline,
        framesep=3mm,
        baselinestretch=1.2,
        bgcolor=light_bg,
        rulecolor=light_rule,
        linenos
    ]{matlab}{AneA/codigos/evaluateObjective.m}
    \caption{Função \texttt{evaluateObjective}}
    \label{code:evaluateobjective}
\end{code}

\subsection{\texttt{spline2.m}}
\label{sec:codigo-spline2}

\begin{code}
    \inputminted[
        frame=leftline,
        framesep=3mm,
        baselinestretch=1.2,
        bgcolor=light_bg,
        rulecolor=light_rule,
        linenos
    ]{matlab}{AneA/codigos/spline2.m}
    \caption{Função \texttt{spline2}}
    \label{code:spline2}
\end{code}


\section{Arquivos de modelo}
\label{sec:arquivos-modelo}

\subsection{\texttt{mainTemplate.m}}
\label{sec:codigo-maintemplate}

\begin{code}
    \inputminted[
        frame=leftline,
        framesep=3mm,
        baselinestretch=1.2,
        bgcolor=light_bg,
        rulecolor=light_rule,
        linenos
    ]{matlab}{AneA/codigos/template/mainTemplate.m}
    \caption{Arquivo principal do template}
    \label{code:maintemplate}
\end{code}

\subsection{\texttt{templateDynamics.m}}
\label{sec:codigo-templatedynamics}

\begin{code}
    \inputminted[
        frame=leftline,
        framesep=3mm,
        baselinestretch=1.2,
        bgcolor=light_bg,
        rulecolor=light_rule,
        linenos
    ]{matlab}{AneA/codigos/template/templateDynamics.m}
    \caption{Função \texttt{templateDynamics} do template}
    \label{code:templatedynamics}
\end{code}


\subsection{\texttt{templateParams.m}}
\label{sec:codigo-templateparams}

\begin{code}
    \inputminted[
        frame=leftline,
        framesep=3mm,
        baselinestretch=1.2,
        bgcolor=light_bg,
        rulecolor=light_rule,
        linenos
    ]{matlab}{AneA/codigos/template/templateParams.m}
    \caption{Função \texttt{templateParams} do template}
    \label{code:templateparams}
\end{code}

\subsection{\texttt{boundaryConstraints.m}}
\label{sec:codigo-boundaryconstraints}

\begin{code}
    \inputminted[
        frame=leftline,
        framesep=3mm,
        baselinestretch=1.2,
        bgcolor=light_bg,
        rulecolor=light_rule,
        linenos
    ]{matlab}{AneA/codigos/template/boundaryConstraints.m}
    \caption{Função \texttt{boundaryConstraints} do template}
    \label{code:boundaryconstraints}
\end{code}


\subsection{\texttt{pathConstraints.m}}
\label{sec:codigo-pathconstraints}

\begin{code}
    \inputminted[
        frame=leftline,
        framesep=3mm,
        baselinestretch=1.2,
        bgcolor=light_bg,
        rulecolor=light_rule,
        linenos
    ]{matlab}{AneA/codigos/template/pathConstraints.m}
    \caption{Função \texttt{pathConstraints} do template}
    \label{code:pathconstraints}
\end{code}


\subsection{\texttt{boundaryObjective.m}}
\label{sec:codigo-boundaryobjective}

\begin{code}
    \inputminted[
        frame=leftline,
        framesep=3mm,
        baselinestretch=1.2,
        bgcolor=light_bg,
        rulecolor=light_rule,
        linenos
    ]{matlab}{AneA/codigos/template/boundaryObjective.m}
    \caption{Função \texttt{boundaryObjective} do template}
    \label{code:boundaryobjective}
\end{code}


\subsection{\texttt{pathObjective.m}}
\label{sec:codigo-pathobjective}

\begin{code}
    \inputminted[
        frame=leftline,
        framesep=3mm,
        baselinestretch=1.2,
        bgcolor=light_bg,
        rulecolor=light_rule,
        linenos
    ]{matlab}{AneA/codigos/template/pathObjective.m}
    \caption{Função \texttt{pathObjective} do template}
    \label{code:pathobjective}
\end{code}


\section{Arquivos do problema movimento simples}
\label{sec:arquivos-problema-movimento-simples}

\subsection{\texttt{mainSimpleMass.m}}
\label{sec:codigo-mainsimplemass}

\begin{code}
    \inputminted[
        frame=leftline,
        framesep=3mm,
        baselinestretch=1.2,
        bgcolor=light_bg,
        rulecolor=light_rule,
        linenos
    ]{matlab}{AneA/codigos/simple_mass/mainSimpleMass.m}
    \caption{Arquivo principal do problema movimento simples}
    \label{code:mainsimplemass}
\end{code}


\subsection{\texttt{simpleMassDynamics.m}}
\label{sec:codigo-simplemassdynamics}

\begin{code}
    \inputminted[
        frame=leftline,
        framesep=3mm,
        baselinestretch=1.2,
        bgcolor=light_bg,
        rulecolor=light_rule,
        linenos
    ]{matlab}{AneA/codigos/simple_mass/simpleMassDynamics.m}
    \caption{Função \texttt{simpleMassDynamics} do problema movimento simples}
    \label{code:simplemassdynamics}
\end{code}


\subsection{\texttt{boundaryConstraints.m}}
\label{sec:codigo-boundaryconstraints-simplemass}

\begin{code}
    \inputminted[
        frame=leftline,
        framesep=3mm,
        baselinestretch=1.2,
        bgcolor=light_bg,
        rulecolor=light_rule,
        linenos
    ]{matlab}{AneA/codigos/simple_mass/boundaryConstraints.m}
    \caption{Função \texttt{boundaryConstraints} do problema movimento simples}
    \label{code:boundaryconstraints-simplemass}
\end{code}


\subsection{\texttt{pathConstraints.m}}
\label{sec:codigo-pathconstraints-simplemass}

\begin{code}
    \inputminted[
        frame=leftline,
        framesep=3mm,
        baselinestretch=1.2,
        bgcolor=light_bg,
        rulecolor=light_rule,
        linenos
    ]{matlab}{AneA/codigos/simple_mass/pathConstraints.m}
    \caption{Função \texttt{pathConstraints} do problema movimento simples}
    \label{code:pathconstraints-simplemass}
\end{code}


\subsection{\texttt{boundaryObjective.m}}
\label{sec:codigo-boundaryobjective-simplemass}

\begin{code}
    \inputminted[
        frame=leftline,
        framesep=3mm,
        baselinestretch=1.2,
        bgcolor=light_bg,
        rulecolor=light_rule,
        linenos
    ]{matlab}{AneA/codigos/simple_mass/boundaryObjective.m}
    \caption{Função \texttt{boundaryObjective} do problema movimento simples}
    \label{code:boundaryobjective-simplemass}
\end{code}


\subsection{\texttt{pathObjective.m}}
\label{sec:codigo-pathobjective-simplemass}

\begin{code}
    \inputminted[
        frame=leftline,
        framesep=3mm,
        baselinestretch=1.2,
        bgcolor=light_bg,
        rulecolor=light_rule,
        linenos
    ]{matlab}{AneA/codigos/simple_mass/pathObjective.m}
    \caption{Função \texttt{pathObjective} do problema movimento simples}
    \label{code:pathobjective-simplemass}
\end{code}


\subsection{\texttt{generateSimpleMassGuess.m}}
\label{sec:codigo-generatesimplemassguess}

\begin{code}
    \inputminted[
        frame=leftline,
        framesep=3mm,
        baselinestretch=1.2,
        bgcolor=light_bg,
        rulecolor=light_rule,
        linenos
    ]{matlab}{AneA/codigos/simple_mass/generateSimpleMassGuess.m}
    \caption{Função \texttt{generateSimpleMassGuess} do problema movimento simples}
    \label{code:generatesimplemassguess}
\end{code}


\section{Arquivos do problema do pêndulo invertido}
\label{sec:arquivos-problema-pendulo-invertido}

\subsection{\texttt{mainPendulumSwingup.m}}
\label{sec:codigo-mainpendulumswingup}

\begin{code}
    \inputminted[
        frame=leftline,
        framesep=3mm,
        baselinestretch=1.2,
        bgcolor=light_bg,
        rulecolor=light_rule,
        linenos
    ]{matlab}{AneA/codigos/pendulum_swingup/mainPendulumSwingup.m}
    \caption{Arquivo principal do problema do pêndulo invertido}
    \label{code:mainpendulumswingup}
\end{code}


\subsection{\texttt{pendulumDynamics.m}}
\label{sec:codigo-pendulumdynamics}

\begin{code}
    \inputminted[
        frame=leftline,
        framesep=3mm,
        baselinestretch=1.2,
        bgcolor=light_bg,
        rulecolor=light_rule,
        linenos
    ]{matlab}{AneA/codigos/pendulum_swingup/pendulumDynamics.m}
    \caption{Função \texttt{pendulumDynamics} do problema do pêndulo invertido}
    \label{code:pendulumdynamics}
\end{code}



\subsection{\texttt{pathObjective.m}}
\label{sec:codigo-pathobjective-pendulum}

\begin{code}
    \inputminted[
        frame=leftline,
        framesep=3mm,
        baselinestretch=1.2,
        bgcolor=light_bg,
        rulecolor=light_rule,
        linenos
    ]{matlab}{AneA/codigos/pendulum_swingup/pathObjective.m}
    \caption{Função \texttt{pathObjective} do problema do pêndulo invertido}
    \label{code:pathobjective-pendulum}
\end{code}


\section{Arquivos do problema da manobra de mudança de faixa}
\label{sec:arquivos-problema-manobra-mudanca-faixa}

\subsection{\texttt{mainLaneChange.m}}
\label{sec:codigo-mainlanechange}

\begin{code}
    \inputminted[
        frame=leftline,
        framesep=3mm,
        baselinestretch=1.2,
        bgcolor=light_bg,
        rulecolor=light_rule,
        linenos
    ]{matlab}{AneA/codigos/lane_change/mainLaneChange.m}
    \caption{Arquivo principal do problema da manobra de mudança de faixa}
    \label{code:mainlanechange}
\end{code}


\subsection{\texttt{carDynamics.m}}
\label{sec:codigo-cardynamics}

\begin{code}
    \inputminted[
        frame=leftline,
        framesep=3mm,
        baselinestretch=1.2,
        bgcolor=light_bg,
        rulecolor=light_rule,
        linenos
    ]{matlab}{AneA/codigos/lane_change/carDynamics.m}
    \caption{Função \texttt{carDynamics} do problema da manobra de mudança de faixa}
    \label{code:cardynamics}
\end{code}


\subsection{\texttt{boundaryConstraints.m}}
\label{sec:codigo-boundaryconstraints-lanechange}

\begin{code}
    \inputminted[
        frame=leftline,
        framesep=3mm,
        baselinestretch=1.2,
        bgcolor=light_bg,
        rulecolor=light_rule,
        linenos
    ]{matlab}{AneA/codigos/lane_change/boundaryConstraints.m}
    \caption{Função \texttt{boundaryConstraints} do problema da manobra de mudança de faixa}
    \label{code:boundaryconstraints-lanechange}
\end{code}


\subsection{\texttt{pathObjective.m}}
\label{sec:codigo-pathobjective-lanechange}

\begin{code}
    \inputminted[
        frame=leftline,
        framesep=3mm,
        baselinestretch=1.2,
        bgcolor=light_bg,
        rulecolor=light_rule,
        linenos
    ]{matlab}{AneA/codigos/lane_change/pathObjective.m}
    \caption{Função \texttt{pathObjective} do problema da manobra de mudança de faixa}
    \label{code:pathobjective-lanechange}
\end{code}


\section{Arquivos do problema da braquistócrona}
\label{sec:arquivos-problema-braquistocrona}

\subsection{\texttt{mainBrachistochrone.m}}
\label{sec:codigo-mainbrachistochrone}

\begin{code}
    \inputminted[
        frame=leftline,
        framesep=3mm,
        baselinestretch=1.2,
        bgcolor=light_bg,
        rulecolor=light_rule,
        linenos
    ]{matlab}{AneA/codigos/brachistochrone/mainBrachistochrone.m}
    \caption{Arquivo principal do problema da braquistócrona}
    \label{code:mainbrachistochrone}
\end{code}


\subsection{\texttt{brachistochroneDynamics.m}}
\label{sec:codigo-brachistochronedynamics}

\begin{code}
    \inputminted[
        frame=leftline,
        framesep=3mm,
        baselinestretch=1.2,
        bgcolor=light_bg,
        rulecolor=light_rule,
        linenos
    ]{matlab}{AneA/codigos/brachistochrone/brachistochroneDynamics.m}
    \caption{Função \texttt{brachistochroneDynamics} do problema da braquistócrona}
    \label{code:brachistochronedynamics}
\end{code}


\subsection{\texttt{brachistochroneParams.m}}
\label{sec:codigo-brachistochroneparams}

\begin{code}
    \inputminted[
        frame=leftline,
        framesep=3mm,
        baselinestretch=1.2,
        bgcolor=light_bg,
        rulecolor=light_rule,
        linenos
    ]{matlab}{AneA/codigos/brachistochrone/brachistochroneParams.m}
    \caption{Função \texttt{brachistochroneParams} do problema da braquistócrona}
    \label{code:brachistochroneparams}
\end{code}


\subsection{\texttt{boundaryConstraints.m}}
\label{sec:codigo-boundaryconstraints-brachistochrone}

\begin{code}
    \inputminted[
        frame=leftline,
        framesep=3mm,
        baselinestretch=1.2,
        bgcolor=light_bg,
        rulecolor=light_rule,
        linenos
    ]{matlab}{AneA/codigos/brachistochrone/boundaryConstraints.m}
    \caption{Função \texttt{boundaryConstraints} do problema da braquistócrona}
    \label{code:boundaryconstraints-brachistochrone}
\end{code}


\subsection{\texttt{pathConstraints.m}}
\label{sec:codigo-pathconstraints-brachistochrone}

\begin{code}
    \inputminted[
        frame=leftline,
        framesep=3mm,
        baselinestretch=1.2,
        bgcolor=light_bg,
        rulecolor=light_rule,
        linenos
    ]{matlab}{AneA/codigos/brachistochrone/pathConstraints.m}
    \caption{Função \texttt{pathConstraints} do problema da braquistócrona}
    \label{code:pathconstraints-brachistochrone}
\end{code}


\subsection{\texttt{boundaryObjective.m}}
\label{sec:codigo-boundaryobjective-brachistochrone}

\begin{code}
    \inputminted[
        frame=leftline,
        framesep=3mm,
        baselinestretch=1.2,
        bgcolor=light_bg,
        rulecolor=light_rule,
        linenos
    ]{matlab}{AneA/codigos/brachistochrone/boundaryObjective.m}
    \caption{Função \texttt{boundaryObjective} do problema da braquistócrona}
    \label{code:boundaryobjective-brachistochrone}
\end{code}


\subsection{\texttt{pathObjective.m}}
\label{sec:codigo-pathobjective-brachistochrone}

\begin{code}
    \inputminted[
        frame=leftline,
        framesep=3mm,
        baselinestretch=1.2,
        bgcolor=light_bg,
        rulecolor=light_rule,
        linenos
    ]{matlab}{AneA/codigos/brachistochrone/pathObjective.m}
    \caption{Função \texttt{pathObjective} do problema da braquistócrona}
    \label{code:pathobjective-brachistochrone}
\end{code}


\subsection{\texttt{generateBrachistochroneGuesses.m}}
\label{sec:codigo-generatebrachistochroneguesses}

\begin{code}
    \inputminted[
        frame=leftline,
        framesep=3mm,
        baselinestretch=1.2,
        bgcolor=light_bg,
        rulecolor=light_rule,
        linenos
    ]{matlab}{AneA/codigos/brachistochrone/generateBrachistochroneGuess.m}
    \caption{Função \texttt{generateBrachistochroneGuess} do problema da braquistócrona}
    \label{code:generatebrachistochroneguess}
\end{code}


\subsection{\texttt{checkConstraints.m}}
\label{sec:codigo-checkconstraints}

\begin{code}
    \inputminted[
        frame=leftline,
        framesep=3mm,
        baselinestretch=1.2,
        bgcolor=light_bg,
        rulecolor=light_rule,
        linenos
    ]{matlab}{AneA/codigos/brachistochrone/checkConstraints.m}
    \caption{Função \texttt{checkConstraints} do problema da braquistócrona}
    \label{code:checkconstraints}
\end{code}


\subsection{\texttt{compareWithAnalytical.m}}
\label{sec:codigo-comparewithanalytical}

\begin{code}
    \inputminted[
        frame=leftline,
        framesep=3mm,
        baselinestretch=1.2,
        bgcolor=light_bg,
        rulecolor=light_rule,
        linenos
    ]{matlab}{AneA/codigos/brachistochrone/compareWithAnalytical.m}
    \caption{Função \texttt{compareWithAnalytical} do problema da braquistócrona}
    \label{code:comparewithanalytical}
\end{code}


\section{Arquivos do problema do eVTOL}
\label{sec:arquivos-problema-evtol}

\subsection{\texttt{mainEvtol.m}}
\label{sec:codigo-mainevtol}

\begin{code}
    \inputminted[
        frame=leftline,
        framesep=3mm,
        baselinestretch=1.2,
        bgcolor=light_bg,
        rulecolor=light_rule,
        linenos
    ]{matlab}{AneA/codigos/evtol/mainEvtol.m}
    \caption{Arquivo principal do problema do eVTOL}
    \label{code:mainevtol}
\end{code}


\subsection{\texttt{evtolDynamics.m}}
\label{sec:codigo-evtoldynamics}

\begin{code}
    \inputminted[
        frame=leftline,
        framesep=3mm,
        baselinestretch=1.2,
        bgcolor=light_bg,
        rulecolor=light_rule,
        linenos
    ]{matlab}{AneA/codigos/evtol/evtolDynamics.m}
    \caption{Função \texttt{evtolDynamics} do problema do eVTOL}
    \label{code:evtoldynamics}
\end{code}


\subsection{\texttt{computeLiftDrag.m}}
\label{sec:codigo-computeliftdrag}

\begin{code}
    \inputminted[
        frame=leftline,
        framesep=3mm,
        baselinestretch=1.2,
        bgcolor=light_bg,
        rulecolor=light_rule,
        linenos
    ]{matlab}{AneA/codigos/evtol/computeLiftDrag.m}
    \caption{Função \texttt{computeLiftDrag} do problema do eVTOL}
    \label{code:computeliftdrag}
\end{code}


\subsection{\texttt{computeInducedVelocity.m}}
\label{sec:codigo-computeinducedvelocity}

\begin{code}
    \inputminted[
        frame=leftline,
        framesep=3mm,
        baselinestretch=1.2,
        bgcolor=light_bg,
        rulecolor=light_rule,
        linenos
    ]{matlab}{AneA/codigos/evtol/computeInducedVelocity.m}
    \caption{Função \texttt{computeInducedVelocity} do problema do eVTOL}
    \label{code:computeinducedvelocity}
\end{code}


\subsection{\texttt{computeFlightAngle.m}}
\label{sec:codigo-computeflightangle}

\begin{code}
    \inputminted[
        frame=leftline,
        framesep=3mm,
        baselinestretch=1.2,
        bgcolor=light_bg,
        rulecolor=light_rule,
        linenos
    ]{matlab}{AneA/codigos/evtol/computeFlightAngle.m}
    \caption{Função \texttt{computeFlightAngle} do problema do eVTOL}
    \label{code:computeflightangle}
\end{code}


\subsection{\texttt{evtolParams.m}}
\label{sec:codigo-evtolparams}

\begin{code}
    \inputminted[
        frame=leftline,
        framesep=3mm,
        baselinestretch=1.2,
        bgcolor=light_bg,
        rulecolor=light_rule,
        linenos
    ]{matlab}{AneA/codigos/evtol/evtolParams.m}
    \caption{Função \texttt{evtolParams} do problema do eVTOL}
    \label{code:evtolparams}
\end{code}


\subsection{\texttt{boundaryConstraints.m}}
\label{sec:codigo-boundaryconstraints-evtol}

\begin{code}
    \inputminted[
        frame=leftline,
        framesep=3mm,
        baselinestretch=1.2,
        bgcolor=light_bg,
        rulecolor=light_rule,
        linenos
    ]{matlab}{AneA/codigos/evtol/boundaryConstraints.m}
    \caption{Função \texttt{boundaryConstraints} do problema do eVTOL}
    \label{code:boundaryconstraints-evtol}
\end{code}


\subsection{\texttt{pathConstraints.m}}
\label{sec:codigo-pathconstraints-evtol}

\begin{code}
    \inputminted[
        frame=leftline,
        framesep=3mm,
        baselinestretch=1.2,
        bgcolor=light_bg,
        rulecolor=light_rule,
        linenos
    ]{matlab}{AneA/codigos/evtol/pathConstraints.m}
    \caption{Função \texttt{pathConstraints} do problema do eVTOL}
    \label{code:pathconstraints-evtol}
\end{code}


\subsection{\texttt{boundaryObjective.m}}
\label{sec:codigo-boundaryobjective-evtol}

\begin{code}
    \inputminted[
        frame=leftline,
        framesep=3mm,
        baselinestretch=1.2,
        bgcolor=light_bg,
        rulecolor=light_rule,
        linenos
    ]{matlab}{AneA/codigos/evtol/boundaryObjective.m}
    \caption{Função \texttt{boundaryObjective} do problema do eVTOL}
    \label{code:boundaryobjective-evtol}
\end{code}


\subsection{\texttt{pathObjective.m}}
\label{sec:codigo-pathobjective-evtol}

\begin{code}
    \inputminted[
        frame=leftline,
        framesep=3mm,
        baselinestretch=1.2,
        bgcolor=light_bg,
        rulecolor=light_rule,
        linenos
    ]{matlab}{AneA/codigos/evtol/pathObjective.m}
    \caption{Função \texttt{pathObjective} do problema do eVTOL}
    \label{code:pathobjective-evtol}
\end{code}


\subsection{\texttt{physicalInitialGuess.m}}
\label{sec:codigo-physicalinitialguess}

\begin{code}
    \inputminted[
        frame=leftline,
        framesep=3mm,
        baselinestretch=1.2,
        bgcolor=light_bg,
        rulecolor=light_rule,
        linenos
    ]{matlab}{AneA/codigos/evtol/physicalInitialGuess.m}
    \caption{Função \texttt{physicalInitialGuess} do problema do eVTOL}
    \label{code:physicalinitialguess}
\end{code}


\subsection{\texttt{checkConstraints.m}}
\label{sec:codigo-checkconstraints-evtol}

\begin{code}
    \inputminted[
        frame=leftline,
        framesep=3mm,
        baselinestretch=1.2,
        bgcolor=light_bg,
        rulecolor=light_rule,
        linenos
    ]{matlab}{AneA/codigos/evtol/checkConstraints.m}
    \caption{Função \texttt{checkConstraints} do problema do eVTOL}
    \label{code:checkconstraints-evtol}
\end{code}

\subsection{\texttt{testAeroForces.m}}
\label{sec:codigo-testaeroforces}

\begin{code}
    \inputminted[
        frame=leftline,
        framesep=3mm,
        baselinestretch=1.2,
        bgcolor=light_bg,
        rulecolor=light_rule,
        linenos
    ]{matlab}{AneA/codigos/evtol/testAeroForces.m}
    \caption{Função \texttt{testAeroForces} do problema do eVTOL}
    \label{code:testaeroforces}
\end{code}


\subsection{\texttt{testEvtolDynamics.m}}
\label{sec:codigo-testevtoldynamics}

\begin{code}
    \inputminted[
        frame=leftline,
        framesep=3mm,
        baselinestretch=1.2,
        bgcolor=light_bg,
        rulecolor=light_rule,
        linenos
    ]{matlab}{AneA/codigos/evtol/testEvtolDynamics.m}
    \caption{Função \texttt{testEvtolDynamics} do problema do eVTOL}
    \label{code:testevtoldynamics}
\end{code}

